\documentclass[journal,12pt,twocolumn]{IEEEtran}
%
\usepackage{setspace}
\usepackage{gensymb}
%\doublespacing
\singlespacing

%\usepackage{graphicx}
%\usepackage{amssymb}
%\usepackage{relsize}
\usepackage[cmex10]{amsmath}
%\usepackage{amsthm}
%\interdisplaylinepenalty=2500
%\savesymbol{iint}
%\usepackage{txfonts}
%\restoresymbol{TXF}{iint}
%\usepackage{wasysym}
\usepackage{amsthm}
%\usepackage{iithtlc}
\usepackage{mathrsfs}
\usepackage{txfonts}
\usepackage{stfloats}
\usepackage{bm}
\usepackage{cite}
\usepackage{cases}
\usepackage{subfig}
%\usepackage{xtab}
\usepackage{longtable}
\usepackage{multirow}
%\usepackage{algorithm}
%\usepackage{algpseudocode}
\usepackage{enumitem}
\usepackage{mathtools}
\usepackage{tikz}
\usepackage{circuitikz}
\usepackage{verbatim}
%\usepackage{tfrupee}
\usepackage[breaklinks=true]{hyperref}
%\usepackage{stmaryrd}
\usepackage{tkz-euclide} % loads  TikZ and tkz-base
\usetkzobj{all}
\usepackage{listings}
    \usepackage{color}                                            %%
    \usepackage{array}                                            %%
    \usepackage{longtable}                                        %%
    \usepackage{calc}                                             %%
    \usepackage{multirow}                                         %%
    \usepackage{hhline}                                           %%
    \usepackage{ifthen}                                           %%
  %optionally (for landscape tables embedded in another document): %%
    \usepackage{lscape}     
\usepackage{multicol}
\usepackage{chngcntr}
%\usepackage{enumerate}

%\usepackage{wasysym}
%\newcounter{MYtempeqncnt}
\DeclareMathOperator*{\Res}{Res}
%\renewcommand{\baselinestretch}{2}
\renewcommand\thesection{\arabic{section}}
\renewcommand\thesubsection{\thesection.\arabic{subsection}}
\renewcommand\thesubsubsection{\thesubsection.\arabic{subsubsection}}

\renewcommand\thesectiondis{\arabic{section}}
\renewcommand\thesubsectiondis{\thesectiondis.\arabic{subsection}}
\renewcommand\thesubsubsectiondis{\thesubsectiondis.\arabic{subsubsection}}

% correct bad hyphenation here
\hyphenation{op-tical net-works semi-conduc-tor}
\def\inputGnumericTable{}                                 %%

\lstset{
%language=C,
frame=single, 
breaklines=true,
columns=fullflexible
}
%\lstset{
%language=tex,
%frame=single, 
%breaklines=true
%}

\begin{document}
%


\newtheorem{theorem}{Theorem}[section]
\newtheorem{problem}{Problem}
\newtheorem{proposition}{Proposition}[section]
\newtheorem{lemma}{Lemma}[section]
\newtheorem{corollary}[theorem]{Corollary}
\newtheorem{example}{Example}[section]
\newtheorem{definition}[problem]{Definition}
%\newtheorem{thm}{Theorem}[section] 
%\newtheorem{defn}[thm]{Definition}
%\newtheorem{algorithm}{Algorithm}[section]
%\newtheorem{cor}{Corollary}
\newcommand{\BEQA}{\begin{eqnarray}}
\newcommand{\EEQA}{\end{eqnarray}}
\newcommand{\define}{\stackrel{\triangle}{=}}

\bibliographystyle{IEEEtran}
%\bibliographystyle{ieeetr}


\providecommand{\mbf}{\mathbf}
\providecommand{\pr}[1]{\ensuremath{\Pr\left(#1\right)}}
\providecommand{\qfunc}[1]{\ensuremath{Q\left(#1\right)}}
\providecommand{\sbrak}[1]{\ensuremath{{}\left[#1\right]}}
\providecommand{\lsbrak}[1]{\ensuremath{{}\left[#1\right.}}
\providecommand{\rsbrak}[1]{\ensuremath{{}\left.#1\right]}}
\providecommand{\brak}[1]{\ensuremath{\left(#1\right)}}
\providecommand{\lbrak}[1]{\ensuremath{\left(#1\right.}}
\providecommand{\rbrak}[1]{\ensuremath{\left.#1\right)}}
\providecommand{\cbrak}[1]{\ensuremath{\left\{#1\right\}}}
\providecommand{\lcbrak}[1]{\ensuremath{\left\{#1\right.}}
\providecommand{\rcbrak}[1]{\ensuremath{\left.#1\right\}}}
\theoremstyle{remark}
\newtheorem{rem}{Remark}
\newcommand{\sgn}{\mathop{\mathrm{sgn}}}
\providecommand{\abs}[1]{\left\vert#1\right\vert}
\providecommand{\res}[1]{\Res\displaylimits_{#1}} 
\providecommand{\norm}[1]{\left\lVert#1\right\rVert}
%\providecommand{\norm}[1]{\lVert#1\rVert}
\providecommand{\mtx}[1]{\mathbf{#1}}
\providecommand{\mean}[1]{E\left[ #1 \right]}
\providecommand{\fourier}{\overset{\mathcal{F}}{ \rightleftharpoons}}
%\providecommand{\hilbert}{\overset{\mathcal{H}}{ \rightleftharpoons}}
\providecommand{\system}{\overset{\mathcal{H}}{ \longleftrightarrow}}
	%\newcommand{\solution}[2]{\textbf{Solution:}{#1}}
\newcommand{\solution}{\noindent \textbf{Solution: }}
\newcommand{\cosec}{\,\text{cosec}\,}
\providecommand{\dec}[2]{\ensuremath{\overset{#1}{\underset{#2}{\gtrless}}}}
\newcommand{\myvec}[1]{\ensuremath{\begin{pmatrix}#1\end{pmatrix}}}
\newcommand{\mydet}[1]{\ensuremath{\begin{vmatrix}#1\end{vmatrix}}}
%\numberwithin{equation}{section}
\numberwithin{equation}{subsection}
%\numberwithin{problem}{section}
%\numberwithin{definition}{section}
\makeatletter
\@addtoreset{figure}{problem}
\makeatother

\let\StandardTheFigure\thefigure
\let\vec\mathbf
%\renewcommand{\thefigure}{\theproblem.\arabic{figure}}
\renewcommand{\thefigure}{\theproblem}
%\setlist[enumerate,1]{before=\renewcommand\theequation{\theenumi.\arabic{equation}}
%\counterwithin{equation}{enumi}


%\renewcommand{\theequation}{\arabic{subsection}.\arabic{equation}}

\def\putbox#1#2#3{\makebox[0in][l]{\makebox[#1][l]{}\raisebox{\baselineskip}[0in][0in]{\raisebox{#2}[0in][0in]{#3}}}}
     \def\rightbox#1{\makebox[0in][r]{#1}}
     \def\centbox#1{\makebox[0in]{#1}}
     \def\topbox#1{\raisebox{-\baselineskip}[0in][0in]{#1}}
     \def\midbox#1{\raisebox{-0.5\baselineskip}[0in][0in]{#1}}

\vspace{3cm}

\title{
%	\logo{
Geometry: JEE Maths
%	}
}
\author{ G V V Sharma$^{*}$% <-this % stops a space
	\thanks{*The author is with the Department
		of Electrical Engineering, Indian Institute of Technology, Hyderabad
		502285 India e-mail:  gadepall@iith.ac.in. All content in this manual is released under GNU GPL.  Free and open source.}
	
}	
%\title{
%	\logo{Matrix Analysis through Octave}{\begin{center}\includegraphics[scale=.24]{tlc}\end{center}}{}{HAMDSP}
%}


% paper titles
% can use linebreaks \\ within to get better formatting as desired
%\title{Matrix Analysis through Octave}
%
%
% author names and IEEE memberships
% note positions of commas and nonbreaking spaces ( ~ ) LaTeX will not break
% a structure at a ~ so this keeps an author's name from being broken across
% two lines.
% use \thanks{} to gain access to the first footnote area
% a separate \thanks must be used for each paragraph as LaTeX2e's \thanks
% was not built to handle multiple paragraphs
%

%\author{<-this % stops a space
%\thanks{}}
%}
% note the % following the last \IEEEmembership and also \thanks - 
% these prevent an unwanted space from occurring between the last author name
% and the end of the author line. i.e., if you had this:
% 
% \author{....lastname \thanks{...} \thanks{...} }s
%                     ^------------^------------^----Do not want these spaces!
%
% a space would be appended to the last name and could cause every name on that
% line to be shifted left slightly. This is one of those "LaTeX things". For
% instance, "\textbf{A} \textbf{B}" will typeset as "A B" not "AB". To get
% "AB" then you have to do: "\textbf{A}\textbf{B}"
% \thanks is no different in this regard, so shield the last } of each \thanks
% that ends a line with a % and do not let a space in before the next \thanks.
% Spaces after \IEEEmembership other than the last one are OK (and needed) as
% you are supposed to have spaces between the names. For what it is worth,
% this is a minor point as most people would not even notice if the said evil
% space somehow managed to creep in.



% The paper headers
%\markboth{Journal of \LaTeX\ Class Files,~Vol.~6, No.~1, January~2007}%
%{Shell \MakeLowercase{\textit{et al.}}: Bare Demo of IEEEtran.cls for Journals}
% The only time the second header will appear i/year/1963s for the odd numbered pages
% after the title page when using the twoside option.
% s
% *** Note that you probably will NOT want to include the author's ***
% *** name in the headers of peer review papers.                   ***
% You can use \ifCLASSOPTIONpeerreview for conditional compilation here if
% you desire.




% If you want to put a publisher's ID mark on the page you can do it like
% this:
%\IEEEpubid{0000--0000/00\$00.00~\copyright~2007 IEEE}
% Remember, if you use this you must call \IEEEpubidadjcol in the second
% column for its text to clear the IEEEpubid ma/year/1963rk.



% make the title area
\maketitle



%\tableofcontents

\bigskip

\renewcommand{\thefigure}{\theenumi}
\renewcommand{\thetable}{\theenumi}
%\renewcommand{\theequation}{\theenumi}

%\begin{abstract}
%%\boldmath
%In this letter, an algorithm for evaluating the exact analytical bit error rate  (BER)  for the piecewise linear (PL) combiner for  multiple relays is presented. Previous results were available only for upto three relays. The algorithm is unique in the sense that  the actual mathematical expressions, that are prohibitively large, need not be explicitly obtained. The diversity gain due to multiple relays is shown through plots of the analytical BER, well supported by simulations. 
%
%\end{abstract}
% IEEEtran.cls defaults to using nonbold math in the Abstract.
% This preserves the distinction between vectors and scalars. However,
% if the journal you are submitting to favors bold math in the abstract,
% then you can use LaTeX's standard command \boldmath ast the very start
% of the abstract to achieve this. Many IEEE journals frown on math
% in the abstract anyway.

% Note that keywords are not normally used for peerreview papers.
%\begin{IEEEkeywords}
%Cooperative diversity, decode and forward, piecewise linear
%\end{IEEEkeywords}



% For peer review papers, you can put extra information on the cover
% page as needed:
% \ifCLASSOPTIONpeerreview
% \begin{center} \bfseries EDICS Category: 3-BBND \end{center}
% \fi
%
% For peerreview papers, this IEEEtran command inserts a page break and
% creates the second title. It will be ignored for othesr modes.
%\IEEEpeerreviewmaketitle


%Download python codes using 
%\begin{lstlisting}
%svn co https://github.com/gadepall/school/trunk/ncert/computation/codes
%\end{lstlisting}

\renewcommand{\theequation}{\theenumi}
\begin{enumerate}[label=\arabic*.,ref=\theenumi]
%\begin{enumerate}[label=\arabic*.,ref=\thesubsection.\theenumi]
\numberwithin{equation}{enumi}
\item Rama was asked by her teacher to subtract 3 from a certain number and then divide the result by 9. Instead, she subtracted 9 and then divided the result by 3. She got 43 as the answer. What would have been her answer if she had solved the problem correctly?

\item A triangle with perimeter 7 has integer side lengths. What is the maximum possible area of such a triangle ?

\item The letters R, M and O represent whole numbers. If $R \times M \times O = 240$, $R \times O + M = 46$, 
$R + O \times M = 64$, What is the value of R + M + O?

\item A postman has to deliver five letters to five different houses. Mischievously, he posts one letter through each door without looking to see if it is the correct address. In how many different ways could he do this so that exactly two of the five houses receive the correct letters ?

\item If 
\begin{align*}
\frac{1}{\sqrt{2011 + \sqrt{2011^2 - 1}}} = \sqrt{m} - \sqrt{n}
\end{align*}
where m and n are positive integers, what is the value of m + n?

\item How many non-negative integral values of x satisfy the equation $[\frac{x}{5}] = [\frac{x}{7}]$ ? (Here [x] denotes the greatest integer less than or equal to x. For example [3.4] = 3 and [-2.3] = -3.)

\item Let N be the set of natural numbers. Suppose $f: N \to {N}$ is a function satisfying the following conditions.
\begin{enumerate}
\item f(m, n) = f(m)f(n);
\item $f(m) < f(n)$ if $m < n$;
\item f(2) = 2. What is the value of $\sum_{k = 1}^{20}$ f(k)?
\end{enumerate}

\item Let AD and BC be the parallel sides of a trapezium ABCD. Let P and Q be the midpoints of the diagonals AC and BD. IF AD  = 16 and BC = 20, what is the length of PQ?

\item In a triangle ABC, let H, I and O be the orthocentre, incentre and circumcentre respectively. If the points B, H, I ,C lie on a circle, what is the magnitude of $\angle BOC$ in degrees?

\item Let ABC be an equilateral triangle. Let P and S be points on AB and AC, respectively, and let Q and R be points on BC such that PQRS is a rectangle. If $PQ = \sqrt{3}PS$ and the area of PQRS is $28\sqrt{3}$, what is the length of PC?

\item Let $A_1$, $B_1$, $C_1$, $D_1$ be the midpoints of the sides of a convex quadrilateral ABCD and let $A_2$, $B_2$, $C_2$, $D_2$ be the midpoints of the sides of the quadriletaral $A_1$, $B_1$, $C_1$, $D_1$. If $A_2$, $B_2$, $C_2$, $D_2$ is a rectangle with sides 4 and 6, then what is the product of the lengths of the diagonals of ABCD?

\item Let S be a circle with centre O. A chord AB, not a diameter, divides S into two regions $R_1$ and $R_2$ such that O belongs to $R_2$. Let $S_1$ be a xircle with centre in $R_1$, touching AB at X and S internally. Let $S_2$ be a circle with centre in $R_2$, touching AB at Y, the circle S internally and passing through the centre of S. The point X lies on the diameter passing through the centre of $S_2$ and $\angle YXO = 30^{o}$. If the radius of $S_2$ is 100, then what is the radius of $S_1$?

\item In a triangle ABC with $\angle BCA = 90^{o}$, the perpendicular bisector of AB intersects segments AB and AC at X and Y, respectively. If the ratio of the area of quadrilateral BXYC to the area of triangle ABC is 13 : 18 and BC = 12 then what is the length of AC?
\item The first term of a sequence is 2014. Each succeeding term is the sum of the cubes of the digits of the previous term. What is the $2014^{th}$ of the sequence?

\item In a triangle with integer side lengths, one side is three times as long as a second side, and the length of the third side is 17. What is the greatest possible perimeter of the triangle? 

\item Let S be a set of real numbers with mean M. If the means of the sets $S \cup \{15\}$ and $S \cup \{15,1\}$ are M+2 and M + 1, respectively, then how many elements does S have?

\item For how many natural numbers n between 1 and 2014 (both inclusive) is $\frac{8n}{9999-n}$ an integer?

\item One morning, each member of Manjul's family drank an 8-ounce mixture of coffee and milk. The amounts of coffee and milk varied from cup to cup, but were never zero. Manjul drank $(\frac{1}{7})^{th}$ of the total amount of milk and 
$(\frac{2}{17})^{th}$ of the total amount of coffee. How many people are there in Manjul's family?

\item Let f be a one-to-one function from the set of natural numbers to itself such that f(mn) = f(m)f(n) for all natural numbers m and n. What is the least possible value of f(999)?

\item Let 
\begin{align*}
x_{1}, x_{2},....,x_{2014}
\end{align*}
be real numbers different from 1, such that 
\begin{align*}
x_{1} + x_{2} +,....,+ x_{2014} = 1
\end{align*} 
\begin{align*}
\frac{x_1}{1 - x_1} + \frac{x_2}{1 - x_2} +...... + \frac{x_{2014}}{1 - x_{2014}} = 1
\end{align*}
What is the value of  
\begin{align*}
\frac{x_1^2}{1 - x_1} + \frac{x_2^2}{1 - x_2} + \frac{x_3^2}{1 - x_3}...... + \frac{x^2_{2014}}{1 - x_{2014}}?
\end{align*}

\item What is the number of ordered pairs (A, B) where A and B are subsets of $\{1,2,...,5\}$ such that neither 
$A \subseteq B$ nor $B \subseteq A$?


\item Positive integers a and b are such that a + b = a/b + b/a. What is the value of $a^2 + b^2$?

\item The equations 
\begin{align*}
x^2 - 4x + k = 0
\end{align*}
\begin{align*}
x^2 + kx - 4 = 0
\end{align*}
where k is a real number, have exactly one common root. What is the value of k?

\item Let P(x) be a non-zero polynomial with integer coefficients. If P(n) is divisible by n for each positive integer n, what is the value of P(0)?

\item Let a, b and c be real numbers such that 
\begin{align*}
a - 7b + 8c = 4 
\end{align*}
\begin{align*}
8a + 4b - c = 7
\end{align*}
What is the value of $a^2 - b^2 + c^2$?

\item If 
\begin{align*}
3^x + 2y =985
\end{align*}
\begin{align*}
3^x - 2^y = 473 
\end{align*}
what is the value of xy?

\item Let a, b and c be such that a + b + c = 0 and 
\begin{align*}
P = \frac{a^2}{2a^2 + bc} + \frac{b^2}{2b^2 + ca} + \frac{c^2}{2c^2 + ab}
\end{align*}
is defined. What is the value of P?


\item In a rectangle ABCD, E is the midpoint of AB; F is a point on AC such that BF is perpendicular to AC; and FE perpendicular to BD. Suppose $BC = \sqrt{3}$, find AB?

\item Suppose in the plane 10 pairwise non-parallel lines intersect one another. What is the maximum possible number of polygons that can be formed?

\item Let P be an interior point of a triangle ABC whose sidelengths are 26, 65, 78. The line through P parallel to BC meets the AB in K and AC in L. The line through P parallel to CA meets BC in M and BA in N. The line through P parallel to AB meets CA in S and CB in T. If KL, MN, ST are of equal lengths, find this common length?

\item Let ABCD be a rectangle and let E and F be points on CD and BC respectively such that are(ADE) = 16, area(CEF) = 9 and area(ABF) = 25. What is the area of triangle AEF?

\item Let AB and CD be two parallel chords in a circle with radius 5 such that the centre O lies between these chords. Suppose AB = 6, CD = 8. Suppose further that the area of the part of the circle lying between the chords AB and CD is 
$m\pi + n/k$, where m, n, k are positive integers with $gcd(m, n, k) = 1$. What is the value of $m + n + k$?

\item Let $\Omega_1$ be a circle with centre O and let AB be a diameter of $\Omega_1$. Let $P$ be a point on the segment OB different from O. Suppose another circle $\Omega_2$ with centre P lies in the interior of $\Omega_1$. Tangents are drawn form A and B to the circle $\Omega_2$ intersectiong $\Omega_1$ again at $A_1$ and $B_1$ respectively such that $A_1$ and $B_1$ are on the opposite sides of AB. Given that $A_1B = 5$, $AB_1 = 15$ and $OP = 10$, find the radius of $\Omega_1$.

\item Consider the areas of the 4 triangles obtained by drawing the diagonals AC and BD of a trapezium ABCD. The product of these areas,taken two at time, are computed. If among the 6 products so obtained, 2 products are 1296 and 576, determine the square root of the maximum possible area of the trapezium to the nearest integer.
\item Let D be an interior point of the side BC of a triangle ABC. Let $I_1$ and $I_2$ be the incentres of triangles ABD and ACD respectively. Let $AI_1$ and $AI_2$ meet BC in E and F respectively. If $\angle BI_1E = 60^{o}$, what is the measure of $\angle CI_2F$ in degrees?

\item Let ABC ba an acute-angled triangle and let H be its orthocentre. Let $G_1$, $G_2$ and $G_3$ be the centroids of the triangles HBC, HCA and HAB respectively. If the area of the triangle $G_1G_2G_3$ is 7 units, what is the area of the triangle ABC?

\item Triangles ABC and DEF are such that $\angle A = \angle D$, AB  = DE = 17, BC = EF = 10  and AC - DF = 12. What is AC + DF?

\item In a triangle ABC, right-angled at A, the altitude through A and the internal bisector of $\angle A$ have lengths 3 and 4 respectively. Find the length of the median through A?

\item In a triangle ABC, the median from B to CA is perpendicular to the median from C to AB. If the median from A to BC is 30, determine $(BC^2 + CA^2 + AB^2)/100$.

\item Let AB be a chord of a circle with centre O. Let C be a point on the circle such that $\angle ABC = 30^{o}$ and O lies inside the triangle ABC. Let D be a point on AB such that $\angle DCO = \angle OCB = 20^{o}$. Find the measure of 
$\angle CDO$ in degrees.

\item Let ABCD be a trapezium in which AB$||$ CD and AD $\perp$ AB. Suppose ABCD has  an incircle which touches AB at Q and CD at P. Given that PC = 36 and QB = 49, find PQ.

\item In a quadrilateral ABCD, it is given that AB = AD = 13, BC = CD = 20, BD = 24. If r is the radius of the circle inscribable, then what is the integer closest to r?
\item Let
\begin{align*}
f(x) = x^2 + ax + b
\end{align*}
If for all non-zero real x
\begin{align*}
f\left(x + \frac{1}{x}\right) = f(x) + f\left(\frac{1}{x}\right)
\end{align*}
and the roots of $f(x) = 0$ are integers, What is the value of $a^2 + b^2$?

\item Let $\overline{abc}$ be a three digit numbers with non-zero digits such that $a^2 + b^2 = c^2$. What is the largest possible prime factor $\overline{abc}$?

\item How many positive integers n are there such that $3 \leq n \leq 100$ and $x^{2^{n}} + x + 1$ is divisible by 
$x^2 + x + 1$?

\item A natutal number $k > 1$ iscalled good if there exist natural numbers
\begin{align*}
a_1 < a_2 < ....a_k
\end{align*}
such that
\begin{align*}
\frac{1}{\sqrt{a_1}} + \frac{1}{\sqrt{a_2}} + ......... + \frac{1}{\sqrt{a_k}} = 1
\end{align*}
Let $f(n)$ be the sum of the first n good numbers, $n \geq 1$. Find the sum of all values of n for which $f(n + 5)/f(n)$ is an integer.

\item Find the number of ordered triples (a, b, c) of positive integers such that
\begin{align*}
30a + 50b + 70c \leq 343
\end{align*}

\item Positive integers x, y, z satisfy $xy + z = 160$. Compute the smallest possible value of $x + yz$?
\end{enumerate}

\end{document}


